% !TEX root = ../main.tex
\newpage
\chapter{Вступ}
\section{Теоретичні дані}
Розглядається одноконтурна система автоматичного цифрового керування (ЦК) з наступною структурною схемою:
\begin{figure}[!ht]
    \begin{circuitikz}[scale = 0.9, transform shape]
        \draw (0, 0) to[switch] (2, 0) node[above left = 0.3em]{$G(s)$} 
        to[lamp] (3, 0) to[switch] (5, 0) node[above left = 0.3em]{$E^*(s)$};
        \node[draw, minimum width = 2cm, minimum height = 1cm, anchor = south west] at (5, -0.5){$W_P^*(s)$};
        \draw (7, 0) to[switch] (9, 0) node[above left = 0.3em]{$u^*(s)$} (9, 0); 
        \node[draw, minimum width = 2cm, minimum height = 1cm, anchor = south west] at (9, -0.5){$W_E^*(s)$};
        \draw (11, 0) to (13, 0);
        \node at (12, 0.6) {$u(s)$};
        \node[draw, minimum width = 2cm, minimum height = 1cm, anchor = south west] at (13,-0.5){$W_O^*(s)$};
        \draw[decoration={markings,mark=at position 1 with
            {\arrow[scale=2,>=stealth]{>}}},postaction={decorate}] (15, 0) -- (17, 0);
        \draw[decoration={markings,mark=at position 1 with
            {\arrow[scale=2,>=stealth]{>}}},postaction={decorate}] (1.5, 0) -- (2.07, 0);
        \draw[decoration={markings,mark=at position 1 with
            {\arrow[scale=2,>=stealth]{>}}},postaction={decorate}] (4.5, 0) -- (5, 0);
        \draw[decoration={markings,mark=at position 1 with
            {\arrow[scale=2,>=stealth]{>}}},postaction={decorate}] (8.5, 0) -- (9, 0);
        \draw[decoration={markings,mark=at position 1 with
            {\arrow[scale=2,>=stealth]{>}}},postaction={decorate}] (12.5, 0) -- (13, 0);
        \draw[decoration={markings,mark=at position 1 with
            {\arrow[scale=2,>=stealth]{>}}},postaction={decorate}] (2.5, -2) -- (2.5, -0.4);
        \draw (2.5, -2) -- (10, -2);
        \draw (10, -2) to[switch] (12, -2) node[above left = 0.3em]{$y^*(s)$}; 
        \node [above = 0.3em] at (16, 0) {$y(s)$};
        \draw (12, -2) -- (16, -2) -- (16, 0);
        \node at (1.7, -0.4) {$+$};
        \node at (2.8, -0.8) {$-$};
        \draw [fill, radius=0.15em] (16, 0) circle;
    \end{circuitikz}
    \caption{Структурна схема типового контура ЦК}
    \label{pic1}
\end{figure}

Тут $W_O(s)$ -- передаточна функція об'єкта керування по керуючому діянню, $G(s)$ і $u(s)$ -- відповідно задаюче і керуюче діяння в формі
перетворення Лапласа, $W_p^*(s)$ -- передаточна функція цифрового регулятора (ЦАП) у формі дискретного перетворення Лапласа,
$W_E(s)$ -- передаточна функція цифро-аналогового регулятора, $E^*(s)$, $u^*(s)$, $y^*(s)$ -- відповідно помилка керування,
керуюче діяння та вихідна керована координата у формі дискретного перетворення Лапласа. Передаточні функції об'єкта для окремих задач мають вигляд
\begin{gather}\label{W_0}
    W_O(s) = \frac{
        k e^{-\tau s}
    }{
        (T_1 s + 1) (T_2 s + 1) (T_3 s + 1)
    } \\
    \label{W_0_12}
    W_O(s) = \frac{
        k (T_1 s + 1)
    }{
        (T_2 s + 1) (T_3 s + 1)
    }
\end{gather}
де $k$ -- коефіцієнт передачі об'єкта керування, $T_1$, $T_2$, $T_3$ -- сталі часу в секундах, $\tau$ -- час запізнення в секундах.

Регулятор ЦК, представлений в різницевій формі на основі позиційного алгоритма пропорційно-інтегрально-диференціального (ПІД)
закону керування записується таким чином:
\begin{gather}\label{PID}
    u(n T_0) = K_p \left(
        e(n T_0) + \frac{T_0}{T_I} \sum_{i=1}^n e(i T_0) +
        \frac{T_D}{T_0} \left[e(n T_0) - e((n-1)T_0)\right]
    \right)
\end{gather}
Тут $u(n T_0)$ та $e(n T_0)$ -- відповідно керуюче діяння і помилка керування в $n$-тий період квантування,
$K_p$ -- коефіцієнт передачі регулятора, $T_I$ та $T_D$ -- відповідно сталі часу інтегрування та диференціювання в секундах,
$T_0$ -- період квантування в секундах. 

Відповідно до \eqref{PID}, дискретна передаточна функція ПІД-регулятора має вигляд
\begin{gather}
    W_p(z) = K_p \left(
        1 + \frac{T_0}{T_I \left(1 - z^{-1}\right)} + \frac{T_D\left(1 - z^{-1}\right)}{T_0}
    \right)
\end{gather}
Якщо час диференціювання $T_D = 0$, то для цифрового ПІ-регулятора матимемо дискретну передаточну функцію
\begin{gather}\label{PI}
    W_p(z) = K_p \left(
        1 + \frac{T_0}{T_I \left(1 - z^{-1}\right)}
    \right)
\end{gather}
де $z = e^{sT_0}$ -- оператор $z$-перетворення.

\section{Завдання курсової роботи}
\begin{enumerate}
    \item Розрахувати дискретну передаточну функцію замкненого контура 
        цифрового керування, попередньо розрахувавши дискретну передаточну 
        функцію приведеної неперервної частини (ПНЧ) об'єкта
        \begin{gather}
            W_\text{п}(z) = z \left\{W_E(s) \cdot W_O(s) \right\}
        \end{gather} 
        для наступних варіантів передаточної функції об'єкта:
        $W_O(S) = \frac{k}{T_1 s + 1}$, $W_O(S) = \frac{k}{(T_1 s + 1)(T_2 s + 1)}$, 
        $W_O(S) = \frac{k e^{-\tau s}}{T_1 s + 1}$, $W_O(S) = \frac{k e^{-\tau s}}{(T_1 s + 1)(T_2 s + 1)}$.
    \item Розрахувати періоди квантування в системі цифрового керування для об'єктів
        \begin{gather}\label{W_01}
            W_{O_1}(s) = \frac{k e^{-\tau s}}{T_1 s + 1} \\
            \label{W_02}
            W_{O_2}(s) = \frac{k e^{-\tau s}}{(T_1 s + 1)(T_2 s + 1)}
        \end{gather}
        і для об'єкта \eqref{W_0_12}, передаточна функція якого має динаміку в чисельнику.
    \item На основі методу <<прямого>> синтезу визначити структуру і оптимальні настройки регуляторів цифрового керування і неперервного
        регулятора для управління об'єктами, передаточні функції яких мають вигляд \eqref{W_01}, \eqref{W_02}. 
        При цьому приймається період квантування $T_0$, розрахований у пункті 2 на основі умови забезпечення необхідної точності керування.
        Значення коефіцієнта підсилення регулятора
        $K_{P_\text{опт}}$ необхідно визначити при таких параметрах настройки $\lambda$:

        a) $\lambda = \frac{1}{T_1}$; б) $\lambda = \frac{1}{1.5 T_1}$; в) $\lambda = \frac{1}{2T_1}$; г) $\lambda = \frac{1}{3T_1}$.

        Для вказаного набору параметрів настройки $\lambda$ шляхом цифрового моделювання побудувати перехідні процеси в замкненому контурі цифрового керування.
    \item  Розрахувати оптимальні параметри ПІ-регулятора цифрового керування і 
        періоду квантування резонансним методом для об'єкта керування \eqref{W_0}, \eqref{W_02}. На основі цифрового моделювання побудувати перехідні процеси 
        вихідної координати $y$ в замкненому контурі при подачі імпульсних тестів 
        на задаюче діяння цифрового регулятора. 
    \item Виконати синтез лінійно-квадратичного регулятора стану і виконати 
        цифрове моделювання замкненої системи з регулятором стану. 
    \item Дослідити стійкість контура цифрового керування, розрахованої за 
        пунктом 3. При цьому використовувати відомі критерії стійкості. 
    \item Сформувати позиційний і швидкісний алгоритм цифрового керування в формі, зручній для програмування для регуляторів цифрового керування відповідно до пунктів 3, 4.
\end{enumerate}

\section{Значення коефіцієнтів та сталих}
\begin{center}
    \begin{tabular}{|c|c|c|c|c|}
        \hline
        $k$ & $T_1$ & $T_2$ & $T_3$ & $\tau$ \\
        \hline
        9.32 & 35 & 19 & 11 & 14 \\ 
        \hline
    \end{tabular}
\end{center}
$k$ -- коефіцієнт передачі об'єкта керування, $T_1$, $T_2$, $T_3$ -- сталі часу в секундах, $\tau$ -- час запізнення в секундах.